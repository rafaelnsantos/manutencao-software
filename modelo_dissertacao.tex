\documentclass[oneside, % para impress�o somente na frante. Oposto a twoside
a4paper,       % tamanho do papel.
12pt,          % tamanho da fonte
brazil		  % o �ltimo idioma � o principal do documento
]{utfprcptex2}

\usepackage[latin1]{inputenc} % pacote para acentuacao direta

\fontetipo{arial}

%%%%%%%% Dados para Capa e outros elementos pretextuais
\instituicao{Universidade Tecnol�gica Federal do Paran�}
\unidade{C�mpus Corn�lio Proc�pio}
\diretoria{Diretoria de Gradua��o e Educa��o Profissional}
\coordenacao{Departamento de Computa��o}
\curso{Engenharia de Software}
\documento{Manuten��o de Software}
\area{Engenharia de Software}
\titulo{T�tulo em Portugu�s} % titulo do trabalho em portugu�s
\autor{Josiel Faleiros, Rafael Santos}
\cita{SOBRENOME, Nome} % sobrenome (mai�sculas), nome do autor do trabalho

\local{Corn�lio Proc�pio} % cidade
\data{\UTFPRanodefesa} % Aqui deve ir o ano da defesa.

\begin{document}
	\capa
	\textual
	
	%---------- Inicio do Texto ----------
	% recomenda-se a escrita de cada capitulo em um arquivo texto separado (exemplo: intro.tex, fund.tex, exper.tex, concl.tex, etc.) e a posterior inclusao dos mesmos no mestre do documento utilizando o comando \input{}, da seguinte forma:
	\input{capitulo1/cap1.tex}
	\input{capitulo2/cap2.tex}
	\input{capitulo3/cap3.tex}
	\input{capitulo4/cap4.tex}
	\input{capitulo5/cap5.tex}
	\input{capitulo6/cap6.tex}
	\input{capitulo7/cap7.tex}
	\input{capitulo8/cap8.tex}
	\input{capitulo9/cap9.tex}
	\input{capitulo10/cap10.tex}
	\input{capitulo11/cap11.tex}
	
	
	%% Formata��o de p�ginas de elementos p�s-textuais
	%\postextual%% N�o comente esta linha
	%---------- Referencias ----------
	%\bibliography{referencias/referencias} % geracao automatica das referencias a partir do arquivo reflatex.bib
	
	%% Gloss�rio
	%\incluirglossario%% Comente para remover este item
	
	%---------- Apendices (opcionais) ----------
	%\apendices
	% Imprime uma p�gina indicando o in�cio dos ap�ndices
	%\partapendices*
	
	%\chapter{Nome do Ap�ndice}
	
	%Use o comando {\ttfamily \textbackslash apendice} e depois comandos {\ttfamily \textbackslash chapter\{\}}
	%para gerar t�tulos de ap�ndices.
	
	%\section{Teste de Se��o em um Ap�ndice}
	
	% ---------- Anexos (opcionais) ----------
	%\anexos
	% Imprime uma p�gina indicando o in�cio dos anexos
	%\partanexos*
	
	
	
	%\chapter{Nome do Anexo}
	
	%\index{Use} Use o comando {\ttfamily \textbackslash anexo} e depois comandos {\ttfamily \textbackslash chapter\{\}}
	%para gerar t�tulos de anexos.
	
	%---------------------------------------------------------------------
	% INDICE REMISSIVO
	%---------------------------------------------------------------------
	%\onecolindex %% Para indice em uma coluna
	\twocolindex  %% Para indice em duas colunas
	\indiceremissivo
	%---------------------------------------------------------------------
	
\end{document} 